%%%%%%%%%%%%%%%%%%%%%%%%%%%%%%%%%%%%%%%%%%%%%%%%%%%%%%%%
%%%%%%%%%%%%%%%%%%%%%%%%%%%%%%%%%%%%%%%%%%%%%%%%%%%%%%%%
\section*{Open Research}
%\openresearch
%%%%%%%%%%%%%%%%%%%%%%%%%%%%%%%%%%%%%%%%%%%%%%%%%%%%%%%%
%%%%%%%%%%%%%%%%%%%%%%%%%%%%%%%%%%%%%%%%%%%%%%%%%%%%%%%%
%
National Water Model (NWM) data used in this study includes the hydrofabric related datasets \cite{nwm2022hydrofabric} including catchments, flowpaths, and reservoirs \cite{nwm2022hydrofabric}.
These are furnished by the National Oceanic and Atmospheric Administration (NOAA) Office of Water Prediction (OWP) via an Earth Science Information Partners (ESIP) Amazon Web Services (AWS) S3 Bucket \cite{esipData2022}.
OWP Flood Inundation Mapping (FIM) capabilities rely extensively on the use of the National Hydrography Plus High Resolution (NHDPlusHR) datasets including BurnLineEvents \cite{nhdplus2022vectors}, value-added attributes (VAA) \cite{nhdplus2022vectors}, water boundaries (WBD) or hydrologic unit code (HUC) geometries \cite{nhdplus2022wbd}, and digital elevation maps (DEM) \cite{nhdplus2022dems}.
Some additional datasets for processing include the National Levee Database (NLD) furnished by the United States Army Core of Engineers (USACE) \cite{engineers2016national}, Land-sea border from the Great Lakes Hydrography Dataset (GLHD) furnished by the Great Lakes Aquatic Habitat Framework (GLAHF) \cite{GreatLakesHydrographyDataset}, and a Land-sea border provided by OpenStreetMap (OSM) \cite{osm2021landsea}.
Evaluation data was furnished by Interagency Flood Risk Management (InFRM) consortium including cross-sections and flood depths \cite{fema2021base,fema2021estimated}.
Additionally, some FIM hydrofabric data including HAND grids, catchments, flowpaths, synthetic rating curves, and cross-walk tables are available on the ESIP bucket \cite{esipData2022}.

Software used in preprocessing data, producing FIM hydrofrabic, generating FIM, computing metrics, and conducting analysis is available from a publicly available Github repository and a HydroShare resource called ``inundation-mapping'' from the ``NOAA-OWP'' organization \cite{inundationMapping2022,imHS2023}.
