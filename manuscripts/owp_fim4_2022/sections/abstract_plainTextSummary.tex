% ------------------------------------------------------------------------ %%
%
%  ABSTRACT and PLAIN LANGUAGE SUMMARY
%
% A good Abstract will begin with a short description of the problem
% being addressed, briefly describe the new data or analyses, then
% briefly states the main conclusion(s) and how they are supported and
% uncertainties.

% The Plain Language Summary should be written for a broad audience,
% including journalists and the science-interested public, that will not have 
% a background in your field.
%
% A Plain Language Summary is required in GRL, JGR: Planets, JGR: Biogeosciences,
% JGR: Oceans, G-Cubed, Reviews of Geophysics, and JAMES.
% see http://sharingscience.agu.org/creating-plain-language-summary/)
%
%% ------------------------------------------------------------------------ %%

%% \begin{abstract} starts the second page

\begin{abstract}
Height Above Nearest Drainage (HAND), a drainage normalizing terrain index, is a means able of producing flood inundation maps (FIMs) from the National Water Model (NWM) at large scales and high resolutions using reach-averaged synthetic rating curves. 
We highlight here that HAND is limited to producing inundation only when sourced from its nearest drainage line, thus lacks the ability to source inundation from multiple fluvial sources.
A version of HAND, known as Generalized Mainstems (GMS), is proposed that discretizes a target stream network into segments of unit Horton-Strahler stream order known as level paths (LP).
The FIMs associated with each independent LP are then mosaiced together, effectively turning the stream network into discrete groups of homogeneous unit stream order by removing the influence of neighboring tributaries.
Improvement in mapping skill is observed by significantly reducing false negatives at river junctions when the inundation extents are compared to FIMs from that of benchmarks.
A more marginal reduction in the false alarm rate is also observed due to a shift introduced in the stage-discharge relationship by increasing the size of the catchments.
We observe that the improvement of this method applied at 4-5\% of the entire stream network to 100\% of the network is about the same magnitude improvement as going from no drainage order reduction to 4-5\% of the network.
This novel contribution is framed in a new open-source implementation that utilizes the latest combination of hydro-conditioning techniques to enforce drainage and counter limitations in the input data.
\end{abstract}
%
\section*{Plain Language Summary}
Flooding is one of the most impactful natural disasters on life and property.
The United States National Water Model (NWM) provides flood forecasts for the entire country so that adequate warnings can be raised to the public to enable safe evacuations and protective measures.
In order to convert flow rates from the NWM to flood inundation maps (FIM), a model, known as Height Above Nearest Drainage (HAND), is used that converts elevation data from height above mean sea-level to height above the nearest river bottom.
This model suffers from issues in mapping performance because inundation sourced from rivers is only considered from the nearest river line.
We developed a technique that mitigates these errors by removing consideration for neighboring tributaries in the relative elevation computation process.
This is done by splitting the stream network into continuous river segments known as level paths (LPs).
These LPs have no tributaries, thus are known to be stream lines with a unit stream order indicating no branching.
HAND is computed independently for each LP and the resulting FIMs are mosaiced together to form one seamless map.
We compared these HAND derived FIMs to maps from physically-based models and found improvement in mapping performance.
%
