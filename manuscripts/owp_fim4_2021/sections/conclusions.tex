%%%%%%%%%%%%%%%%%%%%%%%%%%%%%%%%%%%%%%%%%%%%%%%%%%%%%%%%
%%%%%%%%%%%%%%%%%%%%%%%%%%%%%%%%%%%%%%%%%%%%%%%%%%%%%%%%
\section{Conclusions}
\label{sec:conclusions}
%%%%%%%%%%%%%%%%%%%%%%%%%%%%%%%%%%%%%%%%%%%%%%%%%%%%%%%%
%%%%%%%%%%%%%%%%%%%%%%%%%%%%%%%%%%%%%%%%%%%%%%%%%%%%%%%%
%
Overall, we have produced a product that is able to produce FIMs for the entire US using stream flows whether retrospective or forecast.
Using the HAND method, we can produce this product at very large domains and high resolutions in a relatively short amount of time which enables creating forecast maps at high temporal frequencies.
This yields a significant amount of information that was previously unavailable to forecasters before enabling for higher resolution and higher accuracy forecasts across areas of the country previously unsupported.
All of this is presented in a open-source framework that utilizes some of the latest enhancements to HAND as well as some of the latest data sources available. 
Additionally, a significant issue previously documented with HAND was identified and addressed via the reduction of stream order down to the unary level.
We illustrate how reducing a HAND processing unit to a stream network with unit Horton-Strahler stream order improves FIM skill especially at junctions of higher order and higher flow magnitude rivers.
We additionally illustrate the sensitivity to rating curve bias and implications of stream order reduction.
%
